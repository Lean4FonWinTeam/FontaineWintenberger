\title{Fontaine-Wintenberger Theorem}


% \home{https://lean4fonwinteam.github.io/FontaineWintenberger/}
%\home{localhost:8080}
\home{localhost:8080}
\dochome{localhost:8080/doc}
\github{https://github.com/Lean4FonWinTeam/FontaineWintenberger/}
%\dochome{https://lean4fonwinteam.github.io/FontaineWintenberger/doc/}
%\dochome{localhost:8080/doc}

\maketitle


\tableofcontents

The first three sections serves for the tilting equivalence. The remaining parts serves for the concrete example of cyclotomic fields.

\section{Perfectoid Fields}

\begin{definition}[Perfectoid Fields]
    \label{Perfectoid Field}
    \lean{PerfectoidField}
    % \uses{Valuation.integer}

    Let $K$ be a topological field. We say that $K$ is a \emph{perfectoid field} if and only if the following conditions are satisfied:\\
    (a) the topology on K is induced by a valuation $\| \cdot \| : K → \R _{\ge0}$, and $K$ is complete for this topology;\\
    (b) there exists a non-unit $\pi \in \O_K$ such that $p \in \pi^p \O_K$;\\
    (c) every element of $\O_K/p\O_K$ is a $p$th-power.
\end{definition}



\section{Almost Mathematics}

\section{The Tilting Equivalence for Perfectoid Fields}

\section{Calculation of $\Q_p^{\cyc}$}


\section{Example}
Here you can use LaTex. \cite{marcus}. You must cite something to make it work.


\begin{lemma}\label{List.aux}
        \leanok
        \lean{List.aux}
                this should be a doc string.
    \end{lemma}

\begin{proof}
    \leanok
\end{proof}

\begin{theorem}\label{List.Nat.Ex}
        \leanok
        \lean{List.Nat.Ex}
        \uses{List.aux}
                Here is a doc string
    \end{theorem}

\begin{proof}
    \leanok
\end{proof}

\begin{lemma}\label{List.test}
        \lean{List.test}
                No documentation.
    \end{lemma}

\begin{theorem}\label{List.Test}
        \lean{List.Test}
        \uses{List.test}
                No documentation.
    \end{theorem}


